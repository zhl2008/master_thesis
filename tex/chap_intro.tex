% !TeX root = ../Template.tex
% [绪论]
% 此处为本LaTeX模板的简介
\chapter{绪论}


%%============================
\section{研究背景及意义}
机器学习,数据挖掘等大量数据依赖型的技术在高速发展的同时,对于相关数据的质和量都提出了更高的要求。网络爬虫,又称网络蜘蛛或者网络机器人,在这样一个数据消费的时代,扮演着数据搬运者和传递者的角色。在其运行的生命周期中,往往会按照开发者预设的规则,爬取指定的URL地址或者URL地址列表,并将获取到的数据预处理成标准化的格式[2]。

普通的网络爬虫并没有危害,相反,搜索引擎存储数据的来源都是基于大量分布式网络爬虫爬取得到的结果,网络爬虫在数据传递过程中起到至关重要的重要。但是多线程高并发的失控爬虫,针对网站隐私数据的窃密爬虫,以及不遵守爬虫道德规范的恶意爬虫,都对网络空间健康的生态环境提出了巨大的挑战。

爬虫失控往往是具有多线程操作的通用型爬虫,未能控制爬行时间间隔,或者因为未添加地址环回检测的处理逻辑,在处理特殊的地址链接时陷入死循环中。失控的爬虫通常给网站的性能资源和带宽资源带来巨大的消耗,甚至会影响正常人类用户的体验,这样的爬虫对网站的影响相当于是DDOS攻击[1]。每年的三月份,是失控爬虫的高发期,原因正是因为大量的硕士在写论文时会爬取网站数据用于数据挖掘或者机器学习。

此外,部分互联网公司会爬取其他同行的优质数据用于商业用途,在给其竞争对手带来经济效益的损失的同时,还会促进产业内部恶性竞争的循环。例如,马蜂窝网站的用户评论数据涉嫌造假事件。甚至在一些情况下,恶意爬虫甚至会爬取敏感个人信息用于不法用途,例如某大数据公司非法爬取个人信息被起诉一案。

某些由黑客或者APT组织控制的爬虫,在爬取某些CMS系统或者web中间件的版本信息后,会使用相应的攻击向量攻击脆弱主机[6],给网站服务提供者及使用者造成巨大的损失。
网络空间一直是爬虫与反爬虫战斗的前线,随着反反爬虫技术的不断迭代更新,传统的静态、单一、被动与非实时的反爬虫技术难以与之对抗,或者又因为部署成本和部署带来的性能损失而被束之高阁。

%%============================
\section{国内外研究现状}
  在数据需求不断增加的大数据时代,爬虫技术的发展也日新月异。与此同时,传统单一静态的爬虫识别技术已经无法满足现阶段的需求,研究相关领域的学者也一直在为反爬虫领域提供新的技术和新的思路。

  Guo W等人首先针对传统与单个http 请求进行处理的爬虫识别算法提出新的改进,在他们的文章中,他们首先使用了session粒度的爬虫识别算法,重点关注人类访问session与爬虫访问session中对应的url请求资源类型(样式表,html,图片等)比例不同的特性,并对采集得到的相应的日志进行非实时的线下处理,提取出相关特征作为分类依据。这是最早的基于session的反爬虫机制。
  
  Derek Doran在他们的文章中使用了和Guo W等人类似的方法,他们在使用url请求资源类型比例作为重要特征的同时,引入了离散马尔科夫链的概率模型,并使用该模型得出的log概率来判定访问时来自于人类还是爬虫。但是总体而言,该文章提出的模型并没有过多的创新,而且马尔科夫链的概率模型的计算过程,需要消耗大量的计算资源,不能够应用在实时的爬虫检测上。
 
   与此同时,为了将爬虫识别模型应用到真实的业务场景中,而不仅仅作为一种离线的验证算法。Andoena Balla等人提出了实时的爬虫识别算法。在他们的文章中,他们还引入了如下的session粒度下的特征:(1)head request的百分比 (2)2xx返回的比例 (3)3xx返回的比例 (4)页面资源的访问比例 (5)夜间访问的比例  (6)访问两个页面之间的平均时间  (7)其他二进制文件请求的比例。
   
   Andoena Balla的对session粒度使用了比较完备的特征,为后来基于session粒度的爬虫识别的相关研究,提供了有价值的参考。但对于如何进行session的分类以及如何处理session过长的问题,该文章并没有提供合格的解决方案。Yi Liu在他们的文章中提出了一种解决session过长问题的方法。他们采用了滑动窗口的机制,对每一次处理的http请求做了相关的限制,对在最大程度上保证了处理的http请求的相关性,并在一个窗口内使用SVM来区分普通用户和爬虫。此外,他们还以他们的模型为基础,实现了一整套支持实时爬虫检测的系统。但是该文章在session粒度的特征方面,并没有充分利用session中的包含的信息。
 
  Shengye Wan在他的文章中综合了现有的反爬虫技术,提出一个名为PathMarker的模型。PathMarker会将url地址信息以及当前访问的用户信息加密,并替换掉原有的url地址。由此标注每个请求所对应的session,并利用之前的研究中使用的session特征,来区分爬虫和人类。这种方案可以在很长的时间窗口内持续地追踪爬虫在某个网站的爬行轨迹,可以用于分析爬虫的爬行目标和爬行策略,并且在某种程度上可以追踪使用分布式ip池的爬虫群。 PathMarker在session分类良好的情况下能够产生较好的爬虫检测效果,但是其缺点也很明显,需要修改所有返回请求中的所有链接地址,在真实的生产环境下难以提供灵活的部署方案。此外,一旦访问的爬虫使用对抗性的爬虫策略,PathMarker的session分类效果将不再准确,从而影响最终的爬虫识别效果。

  总体而言,现有的反爬虫技术主要注重于爬虫识别技术的发展,发现可疑爬虫后的阻断方法,往往是单一的阻断或者使用captcha机制进行验证,无法对恶意爬虫的作者起到威慑的作用。在爬虫识别技术的主流技术中,在单粒度上识别的检测技术,一旦遇到采用字段变异的爬虫,便无法发挥应有的作用。而在session粒度上识别的检测技术,在实时性检测上,往往都有较大的性能消耗。除此之外,大部分的爬虫识别模型都没有考虑到爬虫可能采用的反反爬虫手段,对于一些有着特殊对抗策略的爬虫,其识别效果将大打折扣。







% 引用参考\ref{tab:papercomponents}

  %  {\bfseries 论文题目} & \multicolumn{1}{c} {\bfseries 实时性} & \multicolumn{1}{c} {\bfseries 静态信息利用}  &  \multicolumn{1}{c} {\bfseries 动态信息利用}  &  \multicolumn{1}{c} {\bfseries 检测粒度} &  \multicolumn{1}{c} {\bfseries 性能损耗} &  \multicolumn{1}{c} {\bfseries 爬虫对抗与爬虫追踪} \\
\centerline{}
\begin{table}[h]
  \caption{已有研究成果比较}
  \label{tab:papercomponents}
  \centering
\begin{tabular}{|p{5cm}<{\centering}||p{1cm}<{\centering}|p{3cm}<{\centering}|p{1cm}<{\centering}|p{2cm}<{\centering}|p{1cm}<{\centering}|p{1cm}<{\centering}|}
    \hline
    论文题目                                                                                  & 实时性 & 静态信息利用                                                 & 动态信息利用 & 检测粒度        & 性能损耗 & 爬虫对抗 \\
    \hline
    Protecting Web Contents Against Persistent Crawlers                                   & Y   & refer, user agent, cookies                             & N      & session     & 高    & N    \\ 
    \hline
    Web robot detection techniques based on statistics of their requested URL resources   & N   & user agent,URL pattern                                 & N      & 单粒度/session & 高    & N    \\
    \hline
    RESEARCH ON AN ANTI-CRAWLING MECHANISM AND KEY ALGORITHM BASED ON SLIDING TIME WINDOW & Y   & N                                                      & N      & session     & 中    & N    \\ 
    \hline
    Detecting web robots using resource request patterns                                  & N   & N                                                      & N      & session     & 高    & N    \\
    \hline
    Real-time Web Crawler Detection                                                       & Y   & N                                                      & N      & session     & 中    & N    \\
    \hline
    Our paper (Crawler-Net)                                                               & Y   & user agent, http headers value, http headers key order & Y      & 单粒度/session & 中    & Y    \\
    \hline
    \end{tabular}
\end{table}
\centerline{}



\centerline{-----------$\downarrow$-----------Space Check-----------$\downarrow$-----------}
\begin{table}[h]
  \caption{学位论文组成}
  \label{tab:papercomponents}
  \centering
  \begin{tabular}{cp{16\ccwd}p{4cm}}
    \toprule
    {\bfseries 装订顺序} & \multicolumn{1}{c} {\bfseries 内容} & \multicolumn{1}{c} {\bfseries 说明}  \\
    \midrule
    1 & 封面(中、英文)& \\
    2 & 题名页          & \\
    3 & 独创性声明和使用授权书 & \\
    4 & 中文摘要        & \\
    5 & 英文摘要        & \\
    6 & 目录            & \\
    7 & 图表清单及主要符号表  & 根据具体情况可省略 \\
    8 & 主体部分        & \\
    9 & 参考文献        & \\
    10& 附录            & \\
    11&攻读博士学位期间取得的研究成果/ 攻读硕士学位期间取得的学术成果 & 注意博士的是研究成果,硕士的是学术成果 \\
    12& 致谢            & \\
    13& 作者简介        & 硕士学位论文无此项 \\
    \bottomrule
  \end{tabular}
\end{table}
\centerline{-----------$\uparrow$-----------Space Check-----------$\uparrow$-----------}

%%----------------------
\subsection{封面}
\label{sec:error1}

{\bfseries 中图分类号}:根据论文主题内容对照《中国图书分类法》选取;

{\bfseries 论文编号}:北航单位代码(10006)+学号;

{\bfseries 密级}:保密审批通过论文需在封面、题名页直接把相应的“密级☆”及“保密期限”表注在右上角(非密论文务必将相应内容清除),并将《涉密论文审批通知》复印件附在论文最后。密级按由低到高可分为“秘密”、“机密”、“绝密”三级,保密期限可分为“3年”、“5年”、“10年”、“永久”,例如“密级☆ 5年”。鼓励尽量对学位论文进行去密处理;

{\bfseries 学科专业}:以国务院学位委员会批准的授予博士、硕士学位和培养研究生的学科、专业目录中的学科专业为准,一般为二级学科。对专业学位应填相应的工程领域(如航空工程)或专业学位(工商管理硕士)名称;

{\bfseries 指导教师}:以研究生院批准招生的为准,一般只能写一名指导老师,如有经主管部门批准的副指导教师或联合指导老师,可增1名指导教师;

{\bfseries 培养院系}:应准确填写培养的学院或独立系的全称。

%%----------------------
\subsection{题名页}

{\bfseries 研究方向}:只填写一个,应比学科专业的二级学科更具体,但比论文关键词的覆盖面更广,一般为学科分类号对应的研究方向;

{\bfseries 申请学位级别}:学科门类+学位,学科门类有哲学、经济学、法学、教育学、文学、历史学、理学、工学、农学、医学、军事学和管理学等12个学科门类以及专业学位类别(工程、工程管理、公共行政管理、软件工程);

{\bfseries 工作完成日期}:包括学习日期(从研究生入学至毕业时间)、论文提交日期(论文送审评阅时间)、论文答辩日期、学位授予日期;除学位授予日期可以不填外,其他均需准确填写,一律用阿拉伯数字填写日期;

{\bfseries 学位授予单位}:北京航空航天大学。

%%----------------------
\subsection{独创性声明和使用授权书}

必须由作者、指导教师亲笔签名并填写日期。

%%----------------------
\subsection{摘要}

中文摘要包括“摘要”字样,摘要正文及关键词。对于中英文摘要,都必须在摘要的最下方另起一行。

摘要是学位论文内容的简短陈述,应体现论文工作的核心思想。论文摘要应力求语言精炼准确。博士学位论文的中文摘要一般约800$\sim$1200字;硕士学位论文的中文摘要一般约500字。摘要内容应涉及本项科研工作的目的和意义、研究思想和方法、研究成果和结论。博士学位论文必须突出论文的创造性成果,硕士学位论文必须突出论文的新见解。

关键字是为用户查找文献,从文中选取出来揭示全文主体内容的一组词语或术语,应尽量采用词表中的规范词(参考相应的技术术语标准)。关键词一般3$\sim$5个,按词条的外延层次排列(外延大的排在前面)。关键词之间用逗号分开,最后一个关键词后不打标点符号。

为了国际交流的需要,论文必须有英文摘要。英文摘要的内容及关键词应与中文摘要及关键词一致,要符合英语语法,语句通顺,文字流畅。英文和汉语拼音一律为Times New Roman体,字号与中文摘要相同。

%%----------------------
\subsection{目录}

目录按章、节、条和标题编写,一般为二级或三级,目录中应包括绪论(或引言)、论文主体章节、结论、附录、参考文献、附录、攻读学位期间取得的成果等。

%%----------------------
\subsection{图表清单及主要符号表}

如果论文中图表较多,可以分别列出清单置于目录之后。图的清单应有序号、图题和页码,表的清单应有序号、标题和页码。
全文中常用的符号、标志、缩略词、首字母缩写、计量单位、名词、术语等的注释说明,如需汇集,可集中在图和表清单后的主要符号表中列出,符号表排列顺序按英文及其相关文字顺序排出。

%%----------------------
\subsection{主体部分}

一般应包括:绪论(或引言)、正文、结论等部分。

每章应另起一页。章节标题不得使用标点符号,尽量不采用英文缩写词,对必须采用者,应使用本行业的通用缩写词。
三级标题的层次对理工类建议按章(如“第一章”)、节(如“1.1”)、条(如“1.1.1”)的格式编写;对社科、文学类建议按章(如“一、”)、节(如“(一)”)、条(如“1、”)的格式编写,各章题序的阿拉伯数字用Times New Roman字体。

博士学位论文一般为6$\sim$10万字,硕士学位论文一般为3$\sim$5万字。

%%----------------------
\subsection{参考文献}

学术研究应精确、有据、坦诚、创新、积累。而其中精确、有据和积累需要建立在正确对待前人学术成果的基础上。凡有直接引用他人成果之处,均应加标注说明列于参考文献中,以避免论文抄袭现象的发生。

研究生论文参考文献著录及标引按照国家标准《文后参考文献著录规则》(GB774)和中国博硕士学位论文编写与交换格式。

%%----------------------
\subsection{附录}

附录作为论文主体的补充项目,并不是必需的。

%%----------------------
\subsection{成果}

对于博士学位论文,名称用“攻读博士学位期间取得的研究成果”,一般包括:

攻读博士学位期间取得的学术成果:攻读博士学位期间取得的学术成果:列出攻读博士期间发表(含录用)的与学位论文相关的学位论文、发表专利、著作、获奖项目等,书写格式与参考文献格式相同;

攻读博士期间参与的主要科研项目:列出攻读博士学位期间参与的与学位论文相关的主要科研项目,包括项目名称,项目来源,研制时间,本人承担的主要工作。

对于硕士学位论文,名称用“攻读硕士学位期间取得的学术成果”,只列出攻读硕士学位期间发表(含录用)的与学位论文相关的学位论文、发表专利、著作、获奖项目等,书写格式与参考文献格式相同。

%%----------------------
\subsection{致谢}
致谢中主要感谢指导教师在和学术方面对论文的完成有直接贡献及重要帮助的团体和人士,以及感谢给予转载和引用权的资料、图片、文献、研究思想和设想的所有者。致谢中还可以感谢提供研究经费及实验装置的基金会或企业等单位和人士。致谢辞应谦虚诚恳,实事求是,切记浮夸与庸俗之词。

%%----------------------
\subsection{作者简介}

博士学位论文应该提供作者简介,主要包括:姓名、性别、出生年月日、民族、出生的;简要学历、工作经历(职务);以及攻读博士学位期间获得的其他奖项(除攻读学位期间取得的研究成果之外)。
